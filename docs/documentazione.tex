\documentclass{article}
    \usepackage{float}
    \usepackage{textcomp}
    \usepackage{graphicx}
    \usepackage{booktabs}
    \usepackage{color}
    \usepackage{verbatim}
    \usepackage{listings}
    \usepackage{underscore}
    \usepackage{amsmath}
    \usepackage{amssymb}
    %\usepackage{showkeys}
    %\usepackage{float} % here for H placement parameter
    \usepackage{flafter} 
    \setcounter{secnumdepth}{5}
    \usepackage[bookmarks=true]{hyperref}
    \author{\textbf{Marco Bissessur}} 
    \date{23 Giugno 2018}
    \title{ITIS G. Feltrinelli
        \\A.A. 2017\@-\@2018
        \\Documentazione tesina maturità \\ \textbf{Badminton Clubs}
      } 
        \hypersetup{pdftitle={Documentazione},    % title
        pdfauthor={Marco Bissessur},                     % author
        pdfsubject={tesina},                        % subject of the document
        colorlinks=true,       % false: boxed links; true: colored links
        linkcolor=blue,       % color of internal links
        citecolor=blue,       % color of links to bibliography
        filecolor=black,        % color of file links
        urlcolor=purple,        % color of external links
       }
    \begin{document}
    \maketitle
    \begin{center}
        \includegraphics[width=7cm]{logofeltrinelli}
    \end{center}
    \clearpage
    {\hypersetup{hidelinks}\tableofcontents}
    \clearpage
    
    \section{Introduzione}

    \section{Obiettivo}
%L'Obiettivo principale di Badminton Clubs è di aiutare gli utenti a organizzare tornei tra amici o con altri utenti, formare gruppi di amici o incontrare nuove persone attraverso club e  
L'Obiettivo principale di Badminton Clubs è di creare un social network che permette agli utenti di organizzare e gestire tornei tra di loro, nello specifico voglio creare un prodotto che possa:
\begin{itemize}
    \item Permettere agli utenti di inviare richieste di amicizia, accettarle e visuallizare gli amici
    \item Permettere agli utenti di creare tornei con nome, descrizione, numero di partecipanti, tipo di torneo (singolo, doppio) e il sesso dei partecipanti. Gestirli e organizzare i turni.
    \item Permettere agli utenti di creare club e fare gruppi tra di loro.
    \item Permettere agli utenti  di visuallizare i club in base ad una classifica in base al loro punteggio.
    \item Permettere di visualizzare gli utenti iscritti e gli ultimi tornei creati
\end{itemize}
\section{Requisiti funzionali}

    \section{Tecnologia utilizzata}
    \subsection{Front end}
    \subsubsection{Linguaggi}
 I linguaggi Front end che ho utilizzato sono:
 \begin{itemize}
    \item HTML 
    \item CSS 
    \item JAVASCRIPT 
  \end{itemize}
    \subsection{Back end}
    \subsubsection{Linguaggi}
Per programmare il back end ho utilizzato il linguaggio PHP e il linguaggio latex per produrre la Documentazione.
    \subsubsection{Database}
    img ER 
    \subsubsection{}


    \subsubsection{Hosting}
Per gestire il mio sito ho comprato ed utilizzato l'hosting di Aruba, nel quale ho inserito i miei file ed il mio database 
    \section{UX user experience}


    \section{Come ho lavorato}

Per fare il social network ho utilizzato MAMP per lavorare in locale sul mio computer, L'ide che ho utilizzato è Brackets che ho usato per programmare le pagine front e back end. Ho utilizzanto git per salvare e poter tornare ad ogni aggiornamento inserito infine ho inserito il tutto nel mio sito online. \\
Per la Documentazione ho utlizzato il linguaggio Latex e Visual Studio Code come ide.
    

    
    
    \end{document}